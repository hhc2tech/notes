\documentclass[10pt]{article}
\usepackage{vmargin}
\usepackage{calc}


% Set the margins to 1in per proposal request
%
\setpapersize{USletter}
\setmarginsrb{1in}{1in}{1in}{1in}{0pt}{0mm}{0pt}{0mm}
\pagestyle{plain}

% Now set lines per page to 66
%
%\setlength{\textheight}{\baselineskip*65+\topskip}

\begin{document}

\title{Secure Distributed Heterogeneous Systems Management via Confederations and Oracles}
\author{William A. Arbaugh \\
	Department of Computer Science \\ 
	\and 
	Virgil D. Gligor \\
	Department of Electrical Engineering \\ 
	\\ 
	University of Maryland \\ 
	College Park, Maryland 20742 }
\date{}
\maketitle

\section{Introduction}

The tremendous growth rate of the use of information technology has
exacerbated the problem of effectively managing and securing the
resultant information infrastructure. This coupled with the fact that
the current state of the art in security is essentially "penetrate and
patch" has created a situation where information technology is more
vulnerable than ever\cite{Oakland-consensus}. The vulnerability of
information technology is demonstrated by the large number of news
stories relating to wide-spread computer intrusions as well as
controlled network scanning\cite{drogin99}\cite{IAP}. For instance,
the U.S. General Accounting Office released a report detailing
scanning efforts by the U.S. Defense Information Systems Agency
(DISA)\cite{GAO96}. During a three year period (1992-1995), DISA
probed 38,000 different hosts for vulnerabilities finding 65\% of all
hosts vulnerable. While the GAO report did not specify what
vulnerabilities DISA used to attack the military hosts, they likely
used known vulnerabilities.

While previous studies and anecdotal evidence have demonstrated the
increasing vulnerability of information technology, information
security research is currently primarily focused on the underlying
{\it security technology} rather than the secure management of the
information technology. Yet, the tremendous growth in the use of
information technology and it's rate of change creates a configuration
and systems management nightmare that amplifies existing security
problems, and also introduces new security problems as well. Current
approaches for solving this complex problem are {\it ad hoc} and do
not scale well. In this research, we will attack this situation by
conducting a broad examination of distributed heterogeneous
configuration and security management from both a theoretical and a
systems approach. Ensuring that our approach scales and is based upon
a formal representation.

In the remainder of this white paper, we first present our research
objectives followed by our proposed technical approach. Next, the
expected outcome and impact of the research is discussed. Finally, we
conclude the paper.

\section{Research Objectives}

We have three primary research objectives:
\begin{enumerate}

\item XXXXX Virgil change this as needed XXXXX Develop a formal
calculus for representing and reasoning about the configuration of
distributed heterogeneous systems with respect to time.

\item Leveraging the calculus developed above- design and prototype a
system for a scalable and secure distributed heterogeneous systems
management that permits a range of manageability from fully automated
to fully manual. Additionally, the system will permit readiness
information to flow to parent organizations.

\item Investigate technology and methodologies for ensuring the state
of an information system is as expected, i.e. a robust distributed
independent audit capability.

\end{enumerate}

\section{Technical Approach}

The first step in any research project is to understand the problem,
and develop a formal model. The formal model provides the ability to
reason about the domain, and serve as a sound basis for the systems
engineering solution. Finally, once a solution is implemented an
enforcement or auditing process is required to ensure that the system
works as expected. Our technical approach will follow this
methodology. 

\subsection{Formalization of Secure Distributed Hetergeneous Management}
\include{virgil}

\subsection{Methods for Secure Distributed Hetergeneous Management}

XXXXX Discuss concept of confederations and information flows XXXXX

\subsection{Enforcement of Configuration Management}

XXXXX Komoku as an heterogeneous independent auditor XXXXXXXXX

\section{Expected Outcome and Impact of Research}

XXXX Must mention impact on University's Research in support of DOD

XXXX Must mention Impact on University's Teaching

\section{Conclusions}

\appendix

\section{Cost break down by year}
The estimated costs for this research are shown in
Table~\ref{table:costs}.

\begin{table}
\begin{center}
\begin{tabular}{|l|l|}
\hline
Year 1 & \$600,000 \\ \hline
Year 2 & \$600,000 \\ \hline
Year 3 & \$600,000 \\ \hline
Year 4 & \$600,000 \\ \hline
Year 5 & \$600,000 \\ \hline
\end{tabular}
\end{center}
\label{table:costs}
\caption{Estimated costs by year}
\end{table}


\bibliographystyle{ieeetr}
\bibliography{onr}

\end{document}